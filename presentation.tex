\documentclass[12pt]{article}


% - Margin - 1 inch on all sides
\usepackage[letterpaper]{geometry}
\usepackage{times}
\geometry{top=1.0in, bottom=1.0in, left=1.0in, right=1.0in}

% - Double Spacing -
\usepackage{setspace}
\doublespacing

\usepackage{fancyhdr}
\pagestyle{fancy}
\lhead{}
\chead{}
\rhead{Harbour \thepage}
\lfoot{}
\cfoot{}
\rfoot{}
\renewcommand{\headrulewidth}{0pt}
\renewcommand{\footrulewidth}{0pt}
\setlength\headsep{0.333in}


% BEGIN %
\begin{document}
\begin{flushleft}

% Header %
James Harbour \\
Mrs. Dalton \\
Honors English 10/3 \\
8 May 19 \\

% Title %
\begin{center}
Fight or Flight
\end{center}

%Change PG indentation to 0.5in%
\setlength{\parindent}{0.5in}

%% BEGIN PAPER BODY
\normalsize
Six years ago, I took an IQ test. My psychologist confided in my parents about my results, saying that I was mentally handicapped in the processing speed category, and that this inadequacy would affect me my entire life. It was around this time that I began playing Counter-Strike: Global Offensive, a competitive first-person shooter. At this time, my reaction time averaged an abysmal 372 milliseconds. Five years and five-thousand hours later, I was ranked in the 99.25th percentile of players and my reaction time averaged and continues to average around 220 milliseconds. While this improvement is anecdotal, it is not isolated, as esteemed cognitive researcher Daphne Bavelier states in her TED talk on the subject, “action video games have a number of ingredients that are actually really powerful for brain plasticity, learning, attention, vision, etc.” However, society today refuses to reap these benefits as most socially acceptable forms of entertainment promote passivity and indecision. First, let me explain the specifics of this problem.

To many, action video games seem pointless and unnecessary; however this is due to much of the nation's population finding satisfaction in the validity of biased statements about contraversial issues, creating surges of empirically unfounded beliefs. Cognitive researcher Daphne Bavelier in her Ted talk about the subject explains this sentiment: "'Oh come on, can't you do something more intelligent than shooting at zombies?' I'd like to put this kind of knee-jerk reaction in context...There's not one week that goes without some major headlines about wether video games are good or bad for you, right? I'd like to put this kind of Friday night bar discussion aside and get you to actually step into the lab."

To introduce the first area of improvement, consider the massive amount of effort required to switch from doing math homework to writing an APUSH essay. When people require less effort than usual to switch tasks, we say that they have greater cognitive flexibility. The link to violent video games comes from findings in an experment lead by Lorenza Colzato and his colleagues at the Cognitive Psychology Unit and Leiden Institute for Brain and Coginition indicate that "videogame experience is associated with cognitive flexibility...[video game players] showed smaller switcing costs than [non video game players], suggesting that they have better cognitive control skills." Colzato's expirimental procedure accounted for differences in age and IQ, and thus allows the generalization of his results. Thus, action videogames present a concrete strategy for training the elderly "to compensate for losses in their ability to adapt and restructure the cognitive system in response to changing situational demands -- a skill that is essential for almost all 'functional' everyday behavior" (2-3). These results are not singular either, as there are hundreds of new experiments displaying the same results as Colzato. Shifting to the results about vision improvement, a joint experiment by C.S. Green, a researcher at the University of Rochester's Department of Brain and Cognitive Sciences, and the aforementioned Daphne Bavelier concluded that "action video game play enhances subjects' ability in two tasks thought to indicate number of objects apprehensible at a given time" (1). After the initial experiment, followup analysis was performed in the form of a control experiment for the effect of confounding by instantaneous improvement in tracking. The combination of these two experiments establishes causation for playing action video games enhancing the number of objects that can be apprehended and "suggests that this enhancement is mediated by changes in visual short-term memory skills." These skills again counter the passivity of usual modern entertainment by indirectly benefitting brain function. However, the above results were constrained to a

\end{flushleft}
\end{document}
