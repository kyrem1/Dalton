\documentclass[12pt]{article}


% - Margin - 1 inch on all sides
\usepackage[letterpaper]{geometry}
\usepackage{times}
\geometry{top=1.0in, bottom=1.0in, left=1.0in, right=1.0in}

% - Double Spacing -
\usepackage{setspace}
\doublespacing

\usepackage{fancyhdr}
\pagestyle{fancy}
\lhead{}
\chead{}
\rhead{Harbour \thepage}
\lfoot{}
\cfoot{}
\rfoot{}
\renewcommand{\headrulewidth}{0pt}
\renewcommand{\footrulewidth}{0pt}
\setlength\headsep{0.333in}


% BEGIN %
\begin{document}
\begin{flushleft}

% Header %
James Harbour \\
Mrs. Dalton \\
Honors English 10/3 \\
8 May 19 \\

% Title %
\begin{center}
Fight or Flight
\end{center}

%Change PG indentation to 0.5in%
\setlength{\parindent}{0.5in}

%% BEGIN PAPER BODY
\normalsize

%% Introduction %%
Six years ago, I took an IQ test. My psychologist confided in my parents about my results saying that my processing speed was abnormally bad, and that this inadequacy would affect me my entire life. It was around this time that I began playing Counter-Strike: Global Offensive, a competitive first-person shooter. At this time, my reaction time averaged an abysmal 372 milliseconds. Five years and five-thousand hours later, I was ranked in the 99.25th percentile of players and my reaction time averaged and continues to average around 220 milliseconds. While this improvement is anecdotal, it is not isolated, as esteemed cognitive researcher Daphne Bavelier states in her TED talk on the subject, “action video games have a number of ingredients that are actually really powerful for brain plasticity, learning, attention, vision, etc.” However, society today refuses to reap these benefits as most socially acceptable forms of entertainment promote passivity and indecision. First, let me explain the specifics of this problem.

% BPI - Problem %%
To many people, action video games seem pointless and unnecessary; however this is due to much of the nation's population finding satisfaction in the validity of biased statements about contraversial issues, creating surges of empirically unfounded beliefs. Cognitive researcher Daphne Bavelier in her Ted talk about the subject explains this sentiment: "'Oh come on, can't you do something more intelligent than shooting at zombies?' I'd like to put this kind of knee-jerk reaction in context...There's not one week that goes without some major headlines about wether video games are good or bad for you, right? I'd like to put this kind of Friday night bar discussion aside and get you to actually step into the lab."

%% BPII - Solutions %%
To introduce the first area of improvement, consider the massive amount of effort required to switch from doing calculus homework to writing a literary essay.
% LS, Colzato, et. al.
When people require less effort than usual to switch tasks, we say that they have greater cognitive flexibility. The link to violent video games comes from findings in an experment lead by Lorenza Colzato and his colleagues at the Cognitive Psychology Unit and Leiden Institute for Brain and Coginition indicate that "videogame experience is associated with cognitive flexibility...[video game players] showed smaller switcing costs than [non video game players], suggesting that they have better cognitive control skills." Colzato's expirimental procedure accounted for differences in age and IQ, and thus allows the generalization of his results. Thus, action videogames present a concrete strategy for training the elderly "to compensate for losses in their ability to adapt and restructure the cognitive system in response to changing situational demands -- a skill that is essential for almost all 'functional' everyday behavior" (2-3). These results are not singular either, as there are hundreds of new experiments displaying the same results as Colzato.
% C.S. Green and Daphne Bavelier
Shifting to the results about vision improvement, a joint experiment by C.S. Green, a researcher at the University of Rochester's Department of Brain and Cognitive Sciences, and the aforementioned Daphne Bavelier concluded that "action video game play enhances subjects' ability in two tasks thought to indicate number of objects apprehensible at a given time". After the initial experiment, followup analysis was performed in the form of a control experiment for the effect of confounding by instantaneous improvement in tracking. The combination of these two experiments establishes causation for playing action video games enhancing the number of objects that can be apprehended and "suggests that this enhancement is mediated by changes in visual short-term memory skills" (1). These skills again counter the passivity of usual modern entertainment by indirectly benefitting brain function. The above results are constrained to specific tasks which we say indicate cognitive ability, but with the more complex methods of analysis developed in recent years we are able to draw more concrete conclusions.
% C.S. Green and Daphne Bavelier
In a publication presented at the 7th International Conference on Intelligent Human Computer Interaction, Sushil Chandra and his colleagues at the  biomedical engineering department of the Institute of Nuclear Medicine and Allied Science in India through an experiment involving analysis utilizing much more recent and novel techniques than Bayesian statistics such as K-means clustering and linear discriminant analysis along with an EEG find that action videa game training improves general cognitive abilities like reaction time and stress management through more efficient stress reduction (115, 120). These results, occuring many years after the previous studies, truly establish general causation for action video games improving general brain function; a feat not feasible with older methods.
% Franceschini, Sandra, et. al. NB CHANGE WORKS CITED TO Sandra, Franceschini, et. al.
% NB Add page number to citation in works cited
These benefits also have medical implications, as Post-doc Sandra Franceschini and her colleagues at the Universita di Padova through a controlled experiment find that "only 12 hr of playing action video games -- not involving any direct phonological or orthographic training -- drastically [improves] the reading abilities of children with dyslexia" (462). This noteworthy result, published in \emph{Current Biology}, is a testament to the far-reaching and drastic effects of action video games on the brain.

%% BPIII - Counterargument NB ADD CITATIONS BELOW %%
Let me take some time to rebut some of the common arguments against the use of action video games for cognitive improvement. The first of these arguments, used often by parents, is that screentime makes your eyes rot. Although the blue light emitted by screens does interfere with our circadian rhythm if we recieve doses closer to nighttime, the blatent statement of screentime hurting vision completely contradicts results from the aforementioned C.S. Green's experiment and many others, in which action video games actually improve vision, thus contradicting the original claim. The second of these arguments is that action video games are unnecessary violent. Without perusing what constitutes unnecessary violence, the benefits described above, specifically that of improving general function, apply only to players of violent video games as Sushil Chandra's experiment included a control treatment of non-violent video games. The last and most concrete of these arguments is that video games are a waste of time. I will not directly address this argument; however, I would like to indirectly respond by presenting you with a thought excercise.

%% Conclusion %%
Consider all of the time you have spent on the vices of freetime: Netflix, Candy Crush, Youtube, etc. We are all guilty of escaping stress without moving any closer to our terminal goals. What if all of that time had been spent on an action video game? What if every day, when you relax, you were also simultaneously improving your brain's ability? Where would you be? More importantly, where would your brain be? Thank you.
\end{flushleft}
\end{document}
