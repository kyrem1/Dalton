\documentclass[a4paper, 12pt]{article}
\usepackage{graphicx}
\usepackage{fullpage} % changes the margin
\usepackage{enumerate}

\begin{document}
\noindent
\textbf{James Harbour} \\
\normalsize   Due Date: 05/16/2019 \\

\large\textbf{BP1: Problem Explanation}
\normalsize \\ \\
Include Point 0: CS vs. Candy Crush and Television

To many, action video games seem pointless and unnecessary; however this is due to much of the nation's population finding satisfaction in the validity of biased statements about contraversial issues, creating surges of empirically unfounded beliefs. Cognitive researcher Daphne Bavelier in her Ted talk about the subject explains this sentiment: "'Oh come on, can't you do something more intelligent than shooting at zombies?' I'd like to put this kind of knee-jerk reaction in context...There's not one week that goes without some major headlines about wether video games are good or bad for you, right? I'd like to put this kind of Friday night bar discussion aside and get you to actually step into the lab."
\\

\large\textbf{BP2: Solution = Action Video Games}
\normalsize \\

Findings in an experment lead by Lorenza Colzato and his colleagues at the Cognitive Psychology Unit and Leiden Institute for Brain and Coginition indicate that action videogame experience is associated with cognitive flexibility as measured by a task switching paradigm. A blocked design allowed the ruling out of differences attributed to age and IQ. Additionally, playing first-person shooter games predicts performance on a relatively well established test for cognitive flexibility. Thus, action videogames present a concrete strategy to train the elderly to compensate for losses in ability and restructure their brains in response to changing situations, a skill required for almost all "functional" everyday behavoir. \textbf{Turn into true paraphrase.} These results are not singular either, as there are hundreds of new experiments displaying the same results as Colzato. 

\end{document}
